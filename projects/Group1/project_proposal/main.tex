%Author: 
\documentclass[10pt, xcolor=x11names,compress]{beamer}
\usepackage{tabulary}
\usepackage{booktabs}
\usepackage{float}
\usepackage{graphicx}
\usepackage{mwe}% for example pictures
\usepackage{siunitx}
\usepackage{hyperref}

\usetheme{CambridgeUS}
\usecolortheme{beaver}
\useoutertheme{infolines}
\usefonttheme[onlymath]{serif}
\setbeamertemplate{headline}[default]
\setbeamertemplate{navigation symbols}{}
\mode<beamer>{\setbeamertemplate{blocks}[rounded][shadow=true]}
\setbeamercovered{transparent}
\setbeamercolor{block body}{use=structure, fg=white, bg=black!20}
\setbeamercolor{itemize item}{fg=black}
\setbeamercolor{itemize subitem}{fg=gray} 
\setbeamercolor{itemize subsubitem}{fg=black!20} 
\makeatletter\setbeamertemplate{footline}
{  
\leavevmode%  
\hbox{%  
\begin{beamercolorbox}[wd=.333333\paperwidth,ht=2.25ex,dp=1ex,center]{author in head/foot}%    
\usebeamerfont{author in head/foot}
\insertshortauthor%~~\beamer@ifempty{\insertshortinstitute}{}
\insertshortinstitute{Int. MSc. NISER}
 \end{beamercolorbox}%  
 \begin{beamercolorbox}[wd=.333333\paperwidth,ht=2.25ex,dp=1ex,center]{institute in head/foot}%    
 \usebeamerfont{title in head/foot}\insertinstitute  
 \end{beamercolorbox}%  
 \begin{beamercolorbox}[wd=.333333\paperwidth,ht=2.25ex,dp=1ex,right]{date in head/foot}%    
 \usebeamerfont{date in head/foot}\insertshortdate{}\hspace*{2em}    
 \insertframenumber{} / \inserttotalframenumber\hspace*{2ex}   
 \end{beamercolorbox}}%  
 \vskip0pt%
 }
 \makeatother 
 \useoutertheme[footline=empty, subsection=false]{miniframes}
 \usepackage{multicol}  
 
 \begin{document}

\section{Slide 1}
\begin{frame}[label=Background]{\centerline {Prediction of Quantum Dynamics using} \centerline{Experimental Measurements}}
\begin{center}
    \textbf{Abhishek Singh \& Pritipriya Dasbehera}
\end{center}
\begin{itemize}
    \item \underline{\textbf{Idea:}} Predict the evolution of probability density function of a quantum systems from measurements of position.
    \item \underline{\textbf{Dataset:}} We create our own data-set by simulating some theoretical systems like 1-D potential well etc.
    \item \underline{\textbf{Relevant Papers:}}
              \begin{enumerate}
              \scriptsize
                  \item \href{https://doi.org/10.1103/PRXQuantum.4.040337}{\textit{Learning to Predict Arbitrary Quantum Processes} (Dec. 2023)}\\Hsin-Yuan Huang, Sitan Chen, and John Preskill, PRX Quantum 4, 040337
                  \item \href{https://doi.org/10.3389/fmats.2022.1060744}{\textit{Emulating quantum dynamics with neural networks via knowledge distillation} (Jan. 2023)}\\
                  Yu Yao, Chao Cao, Stephan Haas, Mahak Agarwal, Marcin Abram, Front. Mater.,                  Sec. Computational Materials Science
                  \item \href{https://arxiv.org/abs/1711.10566}{\textit{Physics Informed Deep Learning (Part II): Data-driven Discovery of Nonlinear Partial Differential Equations} (Nov. 2017)}\\
                  Maziar Raissi, Paris Perdikaris, George Em Karniadakis, arXiv[cs.AI]
              \end{enumerate}

    
\end{itemize} 
\end{frame}

\section{Slide 2}
\begin{frame}{}
\begin{itemize}
    \item \underline{\textbf{Work Distribution:}}
    \begin{itemize}
        \item{Abhishek:} Literature review on Quantum dynamics, exploring possible representation(s).
        \item{Pritipriya:} Literature review on viable ML algorithms, curating data-set.
        \item{Both:} Slides, Reports, and Implementation of Algorithms.
    \end{itemize}
    \item \underline{\textbf{Algorithms to be implemented:}}
    \begin{itemize}
        \item Surface regression, Physics-Informed Neural Network (PINN). 
    \end{itemize}
    \item \underline{\textbf{Midway Targets:}}
    \begin{itemize}
        \item Curating data-set and organization
        \item Research about possible representation(s).
        \item Implementation, optimization and comparison of 2 or more of above-mentioned algorithms.
    \end{itemize}
        \item \underline{\textbf{Expected Results:}}
    \begin{itemize}
        \item Optimal prediction within the trained region from surface regression.
        \item Optimal prediction outside the trained region from PINN.
    \end{itemize}
\end{itemize}

\end{frame}


\end{document}