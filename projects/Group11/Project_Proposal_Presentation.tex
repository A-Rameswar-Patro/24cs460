%Author: 
\documentclass[10pt, xcolor=x11names,compress]{beamer}
\usepackage{tabulary}
\usepackage{booktabs}
\usepackage{float}
\usepackage{graphicx}
\usepackage{mwe}% for example pictures
\usepackage{siunitx}
\usepackage{hyperref}

\usetheme{Luebeck}
\usecolortheme{orchid}
\useoutertheme{infolines}
\usefonttheme[onlymath]{serif}
\setbeamertemplate{headline}[default]
\setbeamertemplate{itemize items}[square]
\setbeamertemplate{itemize subitem}[triangle]
\setbeamertemplate{navigation symbols}{}
\mode<beamer>{\setbeamertemplate{blocks}[rounded][shadow=true]}
\setbeamercovered{transparent}
\setbeamercolor{block body}{use=structure, fg=white, bg=black!20}
\setbeamercolor{itemize item}{fg=blue}
\setbeamercolor{itemize subitem}{fg=gray} 
\setbeamercolor{itemize subsubitem}{fg=black!20} 
\makeatletter\setbeamertemplate{footline}
{  
\leavevmode%  
\hbox{%  
\begin{beamercolorbox}[wd=.333333\paperwidth,ht=2.25ex,dp=1ex,center]{author in head/foot}%    
\usebeamerfont{author in head/foot}
\insertshortauthor%~~\beamer@ifempty{\insertshortinstitute}{}
\insertshortinstitute{Int. MSc. NISER}
 \end{beamercolorbox}%  
 \begin{beamercolorbox}[wd=.333333\paperwidth,ht=2.25ex,dp=1ex,center]{institute in head/foot}%    
 \usebeamerfont{title in head/foot}\insertinstitute  
 \end{beamercolorbox}%  
 \begin{beamercolorbox}[wd=.333333\paperwidth,ht=2.25ex,dp=1ex,right]{date in head/foot}%    
 \usebeamerfont{date in head/foot}\insertshortdate{}\hspace*{2em}    
 \insertframenumber{} / \inserttotalframenumber\hspace*{2ex}   
 \end{beamercolorbox}}%  
 \vskip0pt%
 }
 \makeatother 
 \useoutertheme[footline=empty, subsection=false]{miniframes}
 \usepackage{multicol}  
 
 \begin{document}

\section{Slide 1}
\begin{frame}[label=Background]{\centerline {Improving ELSA and adding a theoretical framework}}
\begin{center}
    \textbf{A Rameswar Patro \& Aaditya Vicram Saraf (group 11)}
\end{center}
\begin{itemize}
    \item \underline{\textbf{Idea:}} Adding theoretical background to \href{https://openreview.net/forum?id=jFiFmHrIfD}{\textit{Efficient Latent Search Algorithm (ELSA)}} and making improvements to its search components.
    \item \underline{\textbf{Datasets:}} \href{https://www.cs.toronto.edu/~kriz/cifar.html}{CIFAR}, \href{http://yann.lecun.com/exdb/mnist/}{MNIST} and \href{https://www.image-net.org/download.php}{ImageNet-1k}.    
    \item \underline{\textbf{Relevant Papers:}}
              \begin{enumerate}
              \scriptsize
                  \item \href{https://doi.org/10.48550/ARXIV.2105.04906}{\textit{Bardes, A., Ponce, J., \& LeCun, Y. (2021). VICReg: Variance-Invariance-Covariance Regularization for Self-Supervised Learning (Version 3). arXiv.}}
                  \item \href{https://doi.org/10.48550/arXiv.2303.00633}{\textit{Shwartz-Ziv, R., Balestriero, R., Kawaguchi, K., Rudner, T. G. J., \& LeCun, Y. (2023). An Information-Theoretic Perspective on Variance-Invariance-Covariance Regularization (Version 2). arXiv.}}
                  \item \href{https://doi.org/10.48550/arXiv.2208.08795}{\textit {Li, J., Zhou, J., Xiong, Y., Chen, X., \& Chakrabarti, C. (2022). An Adjustable Farthest Point Sampling Method for Approximately-sorted Point Cloud Data (Version 1). arXiv.}}
              \end{enumerate}    
\end{itemize} 
\end{frame}

\section{Slide 2}
\begin{frame}{}
\begin{itemize}
    \item \underline{\textbf{Work Distribution:}}
    \begin{itemize}
        \item{Aaditya :} Understanding Self-Supervised Loss, specifically Contrastive Loss, building mathematical intuition behind the VICReg triplet loss, finding and verifying a theoretical error bound for the loss function, some implementations.
        \item{Rameswar :} Understanding the techniques for Farthest Point Sampling in Graph Neural Networks, adapting Farthest Point Sampling for VICReg Space, some implementations, report. 
    \end{itemize}
    \item \underline{\textbf{Midway Targets:}}
    \begin{itemize}
        \item Complete the reading part of above mentioned topics.
        \item Find some weak error bound for the loss function.
    \end{itemize}
        \item \underline{\textbf{Expected Results:}}
    \begin{itemize}
        \item Improve the working of ELSA by having a better RandS (increasing both the metrics defined in the paper).

        \item Providing theoretical justifications to the working of the two components and finding a theoretical error bound to the said algorithm.
    \end{itemize}
\end{itemize}

\end{frame}


\end{document}



% Understanding Self-Supervised Loss, specifically Contrastive Loss. 
Building mathematical intuition behind the VICReg triplet loss. 

Understanding the techniques for Farthest Point Sampling in Graph Neural Networks.
Adapting Farthest Point Sampling for VICReg Space.

Improving the working of RandS component of ELSA using a better sampling technique to make the two components Orthogonal. 

Finding and verifying a theoretical error bound for the loss function. %
